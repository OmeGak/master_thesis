\documentclass[../main/main.tex]{subfiles}

\begin{document}
\section{Glossary}
\begin{itemize}
\item SED: \hfill spectral energy distribution

\item (spectral) line-force: \hfill force on material in stellar atmosphere

\item LASER: \hfill Light Amplification by Stimulated Emission of Radiation
\end{itemize}

\newpage
\section{General equations - first year overview}

\subsubsection{Hydrodynamics}
Euler equations, together with closing relation (e.g. ideal gas law).

\begin{center}
	\centering
	{\tabulinesep=1.5mm
		\begin{tabu}{|c|c|c|c|} 
			\hline
			\multicolumn{4}{|c|}{primitive variables} \\ \hline
			mass density & velocity & gas energy density & gas pressure \\ 
			$\rho$ & $v$ & $e$ & $p$ \\ \hline
	\end{tabu}}
\end{center}

\subsubsection{Radiation}
Radiative transfer equation: intensity along a ray while interacting with medium. Photons are massless.
\begin{equation}
	\left[ \frac{1}{c}\partial_t + \vec{n}.\vec{\nabla} \right] I_{\nu} =  \eta_{\nu} - \chi_{\nu}I_{\nu}
\end{equation}

\begin{center}
	\centering
	{\tabulinesep=1.5mm
		\begin{tabu}{|c|c|c|c|} 
\hline
frequency & intensity & emissivity & total absorption  \\
$\nu$ & $I_{\nu}$ & $\eta_{\nu}$ & $\chi_{\nu}$ \\ \hline
	\end{tabu}}
\end{center}

These deliver two equations
\begin{itemize}
	\item the radiative energy equation (diffusion flux $\vec{F}$
	\begin{eqnarray}
	\frac{\partial E}{\partial t} + \vec{\nabla} . \vec{F} = \iint ... d\nu d \Omega
	\end{eqnarray}
	\item radiative momentum equation
	\begin{equation}
		\frac{d\vec{F}}{\partial t} = \iint ... \vec{n} d\nu d\Omega
		\end{equation}
\end{itemize}
(after \textbf{integrating over all frequencies}). Depending on the geometry simplifcations, one can e.g. integrate over all solid angles.


\subsubsection{Radiation-Hydrodynamics}
Combination delivers integral-diffusion equation

\begin{eqnarray}
\begin{aligned}
\frac{dI}{d\tau} &=  S - I \\
	&= \int I d\Omega - I
\end{aligned}
\end{eqnarray}

\subsubsection{Challenges}
\begin{itemize}
	\item combination with hydrodynamics
	\item current analysis: simplified geometries (symmetry). E.g. in 2D, an ADI method is used and now also a multigrid method. 
	\item complex geometry difficult to show in ray-tracing scheme
	\item steady-state vs. time dependent
	\item focus on radiation equations
\end{itemize}


\newpage
\section{Very broad introduction \& Summary}
The material here originates from the master thesis of Nicolas Moens \cite{MoensNicolas} and from the course notes \textit{Introduction to numerical methods for radiation in astrophsyics} from professor Sundqvist. 

\subsection{Definitions and equations}

\subsubsection{Radiation equations}
Specific intensity $I(s,\lambda,x,y,t)$ 
\begin{itemize}
\item restrict oursevels to time-independent, one-dimensional (1D) case $I(s,\theta,\lambda)$ where $s$ is the direction of the light ray
%\item defining $I_{\lambda} = \frac{I}{-\chi_{\lambda}}$ 
\item it satisfies Radiation Transfer Equation (RTE) 
$\boxed{\frac{dI_{\lambda}}{d\tau_{\lambda}} = S_{\lambda} - I_{\lambda}}$
\item with 'formal' solution 
$\boxed{I(\lambda,\tau_{\lambda}) = I_0(\lambda) e^{-\tau_{\lambda}}  \int_0^{\tau_{\lambda}}S(t)e^{-t} dt}$
\begin{itemize}
\item no emissivity $S=0$ then $I(\lambda) I_0(\lambda) e^{-\tau_{\lambda}}$
\item no opacity then $I_0(\lambda) = \int_0^{s} \eta_{\lambda}(s)ds$ 
\item constant source function $I(\lambda,\tau) = I_0(\lambda) e^{-\tau_{\lambda}} + S(1-e^{-\tau_{\lambda}})$
\item if $S=a+b\tau$ then $I(\lambda) = a+\frac{b}{k_{\lambda}}$ with $k_{\lambda}$ the opacity. A jump in opacity leads to the jump in intensity of the opposite sign.
\end{itemize}
\end{itemize}

\subsubsection{Radiation equations (bis)}
Material from \cite{TheoryStellarAtmospheres2014}
\begin{equation}
\frac{\delta I(q,t)}{\delta s} = \eta (q,t) - \chi(q,t) I(q,t)
\end{equation}
In cartesian coordinates (with propagation vector $\vec{n} = \left[ \begin{matrix} n_x \\ n_y \\ n_z \end{matrix}  \right]  = \left[ \begin{matrix} \sin(\theta) \cos(\phi) \\ \sin(\theta) \sin(\phi)  \\ \cos(\theta) \end{matrix}  \right]$):
\begin{equation}
\frac{1}{c}\frac{\partial I}{\partial t} + \sin(\theta)\cos(\phi)\frac{\partial I}{\partial x} + \sin(\theta)\sin(\phi)\frac{\partial I}{\partial y} + \cos(\theta) \frac{\partial I}{\partial z} = \eta - \chi I
\end{equation}
\begin{itemize}
\item 1D planar atmosphere: $\frac{\partial I}{\partial x} = \frac{\partial I}{\partial y} = 0$:
\begin{equation}
\frac{1}{c} \frac{\partial I}{\partial t} + \mu \frac{\partial I}{\partial z} = \eta - \chi I
\end{equation}

\item diffusion limit
\end{itemize}

\subsubsection{RHD equations}
The full RHD equations consist of 
\begin{itemize}
\item five partial differential equations
\item one HD closure equation, e.g. (i) variable Eddington tensor method or (ii) flux limited diffusion
\end{itemize}

\paragraph{Heat flux}
The heat flow rate density $\vec{\phi}$ satisfies the Fourier law $\vec{\phi} = - k\nabla T$. More information can be found for instance on \cite{WikiHeat}.

\paragraph{Specific intensity and its angular moments}

\begin{center}
\centering
{\tabulinesep=1.5mm
\begin{tabu}{|c|c|}
\hline 
specific intensity & $\Delta \epsilon = \boxed{I_{\nu}} A_1 A_2/r^2 \Delta \nu \Delta t$ \\ \hline
energy density & $E = \frac{1}{c} \iint I_{\nu} d\nu d\Omega$ \\ \hline
flux vector & $F = \iint I_{\nu}n d\nu d\Omega$ \\ \hline
pressure tensor & $P = \iint I_{\nu} nn d\nu d\Omega$ \\ \hline
mean intensity & $J_{\nu} = \frac{c}{4 \pi} E_{\nu}$ \\ \hline
Eddington flux & $H_{\nu} = \frac{1}{4 \pi} F_{\nu}$ \\ \hline
Eddington's K & $K_{\nu} = \frac{c}{4 \pi}P_{\nu}$ \\ \hline
\end{tabu}}
\end{center}

\paragraph{Eddington factor}
In general, the Eddington factor is a tensor, for 1D systems it is reduced to a scalar.
\begin{equation}
f_{\nu} = \frac{K_{\nu}}{J_{\nu}} = \frac{P_{\nu}}{E_{\nu}}
\end{equation}
\begin{itemize}
\item isotropic radiation field
\item radiation field stronly peaked in radial (i.e. vertical in cartesian) direction
\end{itemize}

\subsubsection{Radiation transport equations, diffusion, equilibrium}
\begin{itemize}
\item black body radiation (Planck function $I_{\nu} = J_{\nu} = B_{\nu}$)
\item in general, extinction(absorption,scattering) and emission
\begin{equation}
\frac{dI_{\nu}}{ds} = j_{\nu} - k_{\nu}I_{\nu}
\end{equation}
\begin{itemize}
\item Cartesian coordinates: 
\begin{equation}
\boxed{\frac{\partial I_{n,\nu}}{\partial t}\frac{1}{c} + n \nabla I_{n,\nu} = j_{\nu} - k_{n,\nu}I_{n,\nu}}
\label{radiation_transfer_equation}
\end{equation}
\item spherical coordinates 
\item 1D-problem with only variation along z-axis $\mu \frac{dI}{dz}  = j -kI$
\item spherical symmetry $\mu \frac{\partial I}{\partial r} + \frac{1-\mu^2}{r}\frac{\partial I}{\partial \mu} = j-kI$
\item plane-parallel approximation 
\begin{equation}
\boxed{\mu \frac{d I}{dr} = j - kI} 
\label{plane_parallel_radiation}
\end{equation}
The angle $\mu$ is constant throughout the computational domain.
Dividing by $k_{\nu}$, this yields 
\begin{equation}
\mu \frac{dI}{k_{\nu} dr} =  \mu \frac{dI}{k_{\nu} dz} = S-I
\end{equation}


\end{itemize}
\item Oth moment equation: integrate Equation (\ref{radiation_transfer_equation}) over $\nu$ and $\Omega$, i.e. $\int d\nu d\Omega$. Conservation of energy
\item first multiply Equation (\ref{radiation_transfer_equation}) with $\frac{n}{c}$ and then do integration
\end{itemize}

\subsubsection{Radiative Diffusion Approximation} The radiative diffusion approximation bridges two regimes: regimes with ... 
\begin{itemize}
\item on one hand, large optical depth $\tau \gg 1$: diffusion equation: temperature structure in a static stellar atmosphere
\item on the other hand, where radiative \textit{transport} is important
\end{itemize}
The diffusive approximation is the following: replace $\boxed{I = B}$ or $I_{\nu} = B_{\nu}$.
\begin{equation}
I_{\nu} = B_{\nu} - \mu \frac{dB_{\nu}}{k_{\nu}dz}
\end{equation}
This equation can be derived as a random walk of photons!


\subsubsection{Applications and approximations for radiative forces}
\begin{itemize}
\item definition of general radiative acceleration vector $g = \frac{1}{\rho c}\int \int n k_{\nu} I_{\nu} d\Omega d\nu$
\begin{itemize}
\item continuum Thomson scattering
\item spectral line with extinction
\begin{itemize}
\item furhtermore assume central continuum source
\item then $g_{line} = \frac{F_{\nu}^0 k_L}{\rho c}$
\end{itemize}
\end{itemize}

\item Sobolev approximation
\item CAK theory
\end{itemize}

\newpage
\subsubsection{Optical depth (recap)}
\paragraph{Optical depth: physical understanding}
Optical depth is the ratio of incident radiant power to transmitted radiant power (\cite{WikiOpticalDepth}).

\begin{center}
\hspace*{-2cm}
\centering
{\tabulinesep=1.5mm
\begin{tabu}{|c|c|c|c|}
\hline 
optical depth 
	& optical depth along ray 
	& line optical depth 
	& Sobolev optical depth \\ \hline
$d\tau = k_{\nu}ds = \sigma_{nu}n ds = \kappa \rho ds$ 
	& $\tau_{\mu,\nu} = \int_z^{z_{max}} \frac{\alpha_{nu}(z')}{\mu}dz' = \frac{\tau_{\nu}(z)}{\mu}$ 
	& $\tau_{\nu} = \int k_L \phi_{\nu} dl = \int \kappa \rho ds$ \\ 
$\tau_{\nu} = \int k_{\nu}ds = \int \sigma_{\nu} n ds$ & \\ \hline
\end{tabu}}
\end{center}
with \begin{itemize}
\item $\sigma$ cross-section
\item $n$ number density
\item $\kappa$ mass absorption density
\item $\rho$ mass density
\item $k_{\nu}$ extinction coefficient
\end{itemize}


\subsection{Overview of symmetry assumptions}
\begin{center}
\centering
{\tabulinesep=1.5mm
\begin{tabu}{|c|c|c|}
\hline 
plane-parallel & 1D atmosphere & \\ 
& bounded by horizontal surfaces & \\ \hline
\end{tabu}}
\end{center}


\subsection{Overview of units}
\begin{center}
\centering
{\tabulinesep=1.5mm
\begin{tabu}{|c|c|}
\hline 
opacity $\alpha = k_{\nu}$ & $\left[ \frac{m^2}{kg} \right] $ \\ \hline

specific intensity $I_{\nu}$ & $\left[ \frac{ergs}{cm^2 . sr . Hz . s} \right] = \left[ \frac{J}{cm^2 . sr . Hz . s} \right] $ \\ \hline

optical depth $\tau$ & \\ 
& $\boxed{\tau = 0}$ leave atmosphere \\ \hline

\end{tabu}}
\end{center}


\subsection{Things to know}
\begin{itemize}
\item expanding flow: redshift (lower frequency)
\item compressing flow: blueshift (higher frequency)
\end{itemize}

\subsection{Definition of specific intensity}
\label{specific_intensity}
The definition of the specific intensity is 
\begin{equation}
I_{\nu} 
	= \frac{dE_{\nu}}{\cos(\theta) d\Omega dt d\nu} 
	= \frac{dE_{\nu}}{\mu d\Omega dt d\nu} 
\end{equation}
On the other hand, for the total energy of a collection of $N$ photons holds that 
\begin{equation}
E_{\nu} = N E_{\nu,\text{photon}} 
\end{equation}

\paragraph{To the point}
From this we deduce that 
\begin{equation}
I_{\nu} \mu  = \frac{N(\mu) dE_{\nu,\text{photon}}}{d\Omega dt d\nu}
\end{equation}
and thus 
\begin{equation}
\boxed{I_{nu} \mu d\mu \sim N(\mu) d\mu}
\end{equation}

\paragraph{Considering the solid angle}
In spherical geometry $d\Omega = \sin(\theta) d\theta d\phi = d\mu d\phi$.



\newpage
\section{Introduction: course material (Sundqvist - CMPAA course)}

\subsection{EXERCISES: Introduction to numerical methods for radiation in astrophysics}
\begin{enumerate}
\item introduction
\item radiation quantities
\begin{itemize}

\item exercise p.3: 
\begin{itemize}
\item on one hand, we know that $\Delta \epsilon \sim C/r^2 $
\item on the other hand, from the definition we know  that $\Delta \epsilon = I_{\nu} A_1 A_2/r^2 \Delta \nu \Delta t$
\item combining these equations shows that $I_{\nu}$ is independent from $r$
\end{itemize}

\item exercise p.4:
\begin{itemize}
\item 
\end{itemize}

\item exercise 1:
\begin{itemize}
\item $F_x =  \int_0^{\pi} \left[ I_{\nu}(\theta)\sin^2(\theta) \int_0^{2 \pi}\cos(\phi) \right] d\theta d \phi = 0 $
\item the same reasoning for $F_y = 0$
\end{itemize}

\item exercise 2:
\begin{itemize}
\item the equation follows from $d\mu = d\cos(\theta) = \sin(\theta) d\theta$
\end{itemize}

\item exercise 3: 
\begin{itemize}
\item isotropic radiation field (i.e. $I(\mu) = I$) then we have $F_{\nu} = 2 \pi  \int_{-1}^{1} I \mu d\mu = 2 \pi I \left. \frac{x^2}{2}\right \rvert_
{-1}^{1} = 0$
\end{itemize}

\item exercise 4:
\begin{itemize}
\item $F_{\nu} = 2 \pi  \int_{-1}^{1} I(\mu) \mu d\mu 
	= 2 \pi  \int_{-1}^{0} I_{\nu}^{-} \mu d\mu 
	+ 2 \pi  \int_{0}^{1} I_{\nu}^{+} \mu d\mu 
	= 2 \pi I_{\nu}^+ $
\end{itemize}

\item exercise p.7:
\begin{itemize}

\item isotropic radiation field:
\begin{itemize}
\item although the radiation pressure is a tensor, we will denote it as a scalar $P_{\nu} = \frac{4 \pi I_{\nu}}{c}$
\item the radiation energy density $E_{\nu} = \frac{12 \pi I_{\nu}}{c} $
\item thus $f_{\nu} = \frac{1}{3}$
\end{itemize}

\item very strongly peaked in radial direction (beam): $I_{\nu} = I_0 \delta(\mu- \mu_0)$ with $\mu_0 = 1$
\begin{itemize}
\item pressure tensor $P_{nu} = \frac{1}{c} \int I_0 \delta(\mu- \mu_0) nn d \Omega$
\item energy density $E_{\nu} = \frac{1}{c}\int I_{\nu} d\Omega$
\item in this case $P_{\nu} = E_{\nu}$ thus $f_{\nu} = 1$
\end{itemize}
\end{itemize}

\end{itemize}

\item radiation transport vs. diffusion vs. equilibrium
\begin{itemize}
\item exercise p. 12: 1D, Cartesian geometry, plane-parallel, frequency-independent and isotropic emission/extinction
\begin{itemize}

\item radiation energy equation 
\begin{itemize}
\item The equation follows by integrating Equation (\ref{plane_parallel_radiation})
\item By definition, $E= \frac{1}{c}\iint I_{\nu} d\nu d\Omega$
\item thus $\frac{dE}{dr} = \int (j-kI) d\nu d\Omega$ thus $\boxed{\frac{dE}{dr} = \frac{(j-kI)4\pi(\nu_1-\nu_0)}{c}}$
\item work out the integral taking into account frequency-independent and isotropic coefficients: 
\end{itemize}

\item zeroth momentum equations
\begin{itemize}
\item One must also take into account the specific form of the flux vector \\ $F = \iint I_{\nu} n d\nu d\Omega = 2 \pi \int_{-1}^1 I_{\nu}(\mu) \mu d \mu$
\item thus $\frac{dF}{dr} = \frac{1}{c} \int (j-kI) n d\nu d\Omega$ thus $\boxed{\frac{dF}{dr} = \frac{(j-kI)4\pi(\nu_1-\nu_0)n}{c}}$
\end{itemize} 

\item first moment equation
\begin{itemize}
\item similar reasoning
\item $\frac{dP}{dr} = \int (j-kI) n.n d\nu d\Omega$ thus $\boxed{\frac{dF}{dr} = \frac{(j-kI)4\pi(\nu_1-\nu_0)n}{c}}$
\end{itemize}
\end{itemize}

\item first exercise p. 15
\begin{itemize}
\item $P = \frac{1}{c}\iint I_{\nu} \mu^2 d\Omega d\nu = \frac{2 \pi}{c} \int_{\nu} \int_{-1}^{1} I_{\nu} \mu^2 d\mu d\nu = \frac{4 \pi}{3c} \int B_{\nu} d\nu = \frac{a T^4}{3} = \frac{E}{3} $
\end{itemize}

\item second exercise p.15 
\begin{itemize}
\item assuming the diffusion limit, 
\item flux-weighted mean opacity $\kappa_F = \frac{\int F_{\nu} \kappa_{\nu}d\nu}{\int F_{\nu} d\nu}$
\item Rosseland mean opacity $\frac{1}{\kappa_R} = \frac{\int_0^{\infty}\frac{1}{\kappa_{\nu}}\frac{dB_{\nu}}{dT}}{\int_0^{\infty} \frac{dB_{\nu}}{dT} d\nu}$. 
\begin{itemize}
\item in the diffusion limit, $F_{\nu} = - \frac{4 \pi}{3}\frac{d B_{\nu}}{k_{\nu} dz}$ thus $\frac{dB_{nu}}{dT} =$
\item  
\end{itemize}
\end{itemize}


\item third exercise p.15

\end{itemize}



\item the equations of radiation-hydrodynamics
\item numerical techniques for the radiative diffusion approximation

\item applications and approximations for a dynamically important radiative force in supersonic flows
\begin{itemize}
\item exercise p.27: $L_{SOB} = \Delta r = \frac{v_{th}}{dv/dr} = \frac{10 [km/s]}{1000 [km/s]/R_{*}} = 0.01 R_{*}$
\end{itemize}

\item Appendix A: properties of equilibrium black-body radiation
\begin{itemize}

\item exercise p. 29
\begin{itemize}
\item this should be satisfied: $B_{\nu} d\nu = -B_{\lambda} d\lambda$ and also $\nu  = \frac{c}{\lambda}$
\item this is equivalent to saying that $0 = \nu d \lambda + \lambda d \nu$ or $d \lambda = - \frac{\lambda}{\nu} d\nu$ thus $B_{\lambda} = \frac{\nu}{\lambda} B_{\nu}$
\item $B_{\lambda}(T) = \frac{\nu}{\lambda} \frac{2h \nu^3}{(\lambda \nu)^2} \frac{1}{e^{h c/ \lambda kT} - 1} = \frac{2h \nu^2}{\lambda^3} \frac{1}{e^{h c/ \lambda kT} - 1} = \frac{2hc^2}{\lambda^5} \frac{1}{e^{h c/ \lambda kT} - 1}$ 
\end{itemize}

\item first exercise p.31
\begin{itemize}
\item derive that $\lambda_{max} T = 2897.8 [\mu m K]$
\item ...
\end{itemize}

\item second exercise p.31
\begin{itemize}
\item this is about the spectra of (unknown) stars
\end{itemize}

\item first exercise p.32
\begin{itemize}
\item see exercise 7
\end{itemize}

\item second exercise p.32
\begin{itemize}
\item BB radiation: $I_{\nu} = \frac{2h \nu^3}{c^2} \frac{1}{e^{h \nu/kt}-1}$
\item the radiative flux for isotropic BB radiation is zero. See also exercise 3. This dus also holds for BB radiation.
\end{itemize}

\item exercise p. 33
\begin{itemize}
\item \underline{HR-diagram}
\end{itemize}

\end{itemize}

\item Appendix B: Simple examples to the radiative transfer equation
\begin{itemize}

\item first exercise p. 34
\begin{itemize}
\item start from radiative transport equation $\mu \frac{dI}{ds} = \alpha - \eta I$ in which $\eta = 0$ thus $\boxed{\mu \frac{dI}{ds} = \alpha}$
\item solving the ODE in the general case that $\alpha(s)$ is not constant: 
\begin{itemize}
\item integrate the equation $\mu I = \int_0^D \alpha ds$
\item ...
\end{itemize}
 
\item second exercise p. 34
\begin{itemize}
\item case $\tau(D) >> 1$: then $I(D) \approx S$ 
\item case $\tau(D) << 1$: then $I(D) \approx I(0)+S(1-1) = I(0)$
\end{itemize} 
 
\item first exercise p.35
\begin{itemize}
\item is the plane-parallel approximation valid for the solar photosphere?
\end{itemize} 

\item second exercise p.35
\begin{itemize}
\item goal: find a solution to the equation $\mu \frac{dI_{\nu}}{d\tau_{\nu}} = I_{\nu} - S_{\nu}$ where $I(\tau,\mu)$
\item solution
\end{itemize}
 
\end{itemize}

\item second exercise p.35
\end{itemize}

\item Appendix C: connecting random walk of photons with radiative diffusion model
\begin{itemize}
\item exercise p. 38. Computing the average photon mean-free path inside the Sun. \\
$l = \frac{1}{\kappa \rho} = \frac{V_o}{\kappa M_o} [cm]$

\item exercise p.39. Computing the random-walk time (diffusion time) for photons

\end{itemize}


\end{enumerate}

\subsection{Implicit 1D solver (20-11-2018)}
See computer code
\subsection{ADI 2D Solver}
See computer code
\subsection{Area of a circle}
See computer code
\subsection{Limb Darkening}
See Section \ref{limb_darkening_discussion}.


\newpage
\section{The mathematics of Radiative Transfer}
The material in this section is based on the book \cite{Busbridge}.

\subsection{Auxiliary mathematics}

\begin{itemize}
\item $\cos(\Theta) = cos(\theta)\cos(\theta') + \sin(\theta)\sin(\theta') \cos(\phi-\phi')$

\item phase function $\boxed{p(\mu,\phi,\mu',\phi',\tau) = \sum_{n=0}^N \omega_n P_n(\cos(\Theta))}$
\begin{itemize}

\item isotropic scattering $p(\tau) = \omega_0(\tau)$
\end{itemize}

\item equation of transfer $\boxed{\mu \frac{\partial I(\tau,\mu,\phi)}{\partial \tau} = I(\tau,\mu,\phi) - \mathcal{S}(\tau,\mu,\phi)}$ 
\\ with $\mathcal{S}(\tau,\mu,\phi) = B_1(\tau) + \frac{1}{4 \pi} \int_{-1}^{1} d\mu' \int_0^{2\pi} I(\tau,\mu',\phi') p(\mu,\phi,\mu',\phi') d\phi'$
\begin{itemize}
\item axially symmetric with isotropic scattering \\
$\mathcal{S}(\tau) = \frac{\omega_0(\tau)}{2} \int_{-1}^{1} I(\tau,\mu') d\mu' = 
B_1(\tau) + \frac{\omega_0(\tau)}{2} \int_0^{\tau_1} \mathcal{S} (t) E_1(|t-\tau|)dt $
\item the Milne equation of the problem $(1-\omega_0 \bar{\Lambda})\{ \\mahtcal{S}(t)\} = B(\tau)$
\begin{itemize}
\item solve for $\mathcal{S}(t)$
\item then find $I(\tau,\mu)$
\end{itemize}

\end{itemize}
\end{itemize}

\subsection{The H-functions}
\begin{itemize}
\item characteristic equation
\end{itemize}

\newpage
\section{Monte Carlo and Radiative Transfer (Puls)}
\subsection{basic definitions and facts}
\subsection{about random numbers}
\subsection{MC integration}
\subsection{MC simulation}
\paragraph{Radiative transfer in stellar atmospheres}
\begin{itemize}
\item GOAL: spatial radiation energy density $E(\tau)$ in an atmospheric layer 
\begin{itemize}
\item only photon-electron scattering
\item $\tau$ is the optical depth
\end{itemize}

\item Milne's integral equation $\boxed{E(\tau) = \frac{1}{2} \int_0^{\infty} E(t) E_1(|t-\tau|) dt}$
\begin{itemize}
\item analytical solution $\frac{E(\tau)}{E(0)} = \sqrt{3} (\tau + q(\tau))$
\item MC simulation
\begin{itemize}
\item emission angle
\item optical depth until next scattering event
\item scattering angle
\end{itemize}
\end{itemize}

\item HOW DOES THIS WORK?
\end{itemize}

\begin{algorithm}
\caption{Limb darkening: compute quantitiy of photons}\label{limb_darkening}
\begin{algorithmic}
\State create photons

\State probability distribution for emission angle $\mu = \cos(\theta)$: $\boxed{p(\mu) d \mu = \mu d \mu}$

\State optical depth until next scattering event: $\boxed{p(\tau)dt \approx e^{-\tau} d\tau}$

\State isotropic scattering angle at low energies: $\boxed{p(\mu) d\mu \approx d\mu}$

\State follow all photons until they leave the atmosphere or are scattered back into stellar interior
\end{algorithmic}
\end{algorithm}

\subsection{Exercise 1: RNG}
\subsection{Exercise 2: Planck-function}
\begin{enumerate}
\item analytical method
\item MC method
\end{enumerate}
\subsection{limb darkening}
See section \ref{limb_darkening_discussion}.

\newpage

\newpage
\section{Introduction to Monte Carlo Radiation Transfer (Wood+)}
The material is taken from
\begin{itemize}
\item (Wood, Wittney, Bjorkman, Wolff - 2001)
\item (Wood, Wittney, Bjorkman, Wolff - 2013)
\end{itemize}

\subsection{Elementary principles}

\begin{center}
\centering
{\tabulinesep=1.5mm
\begin{tabu}{|c|c|}
\hline 
specific intensity & $I_{\nu}$ \\ \hline
radiant energy & $dE_{\nu}$ \\ \hline
surface area & $dA$ \\ \hline
angle & $\theta$ \\ \hline
solid angle & $d \Omega$ \\ \hline
frequency range & $d \nu$ \\ \hline
time & $dt$ \\ \hline
flux & $F_{\nu}$ \\ \hline
cross section & $\sigma$ \\ \hline
scattering angle & $\chi$ \\ 
 & $\mu = \cos(\chi)$ \\ \hline
mean intensity & $J$ \\ \hline
flux & $H$ \\ \hline
radiation pressure & $K$ \\ \hline
\end{tabu}}
\end{center}


\begin{center}
\centering
{\tabulinesep=1.5mm
\begin{tabu}{|c|c|}
\hline 
intensity & $I_{\nu}(l) = I_{\nu}(0)e^{n \sigma l}$ \\ \hline
angular phase function of the scattering particle & $P(\cos(\chi))$ \\ \hline

\end{tabu}}
\end{center}

\begin{center}
\centering
{\tabulinesep=1.5mm
\begin{tabu}{|c|c|}
\hline 
inverse method & $\xi = \int_0 ^{x_0} P(x) dx $ with $\xi \in \mathcal{U}(0,1)$ \\ \hline

rejection method & \\ \hline
\end{tabu}}
\end{center}

\subsection{Eddington factors}

\subsection{Example: plane parallel atmosphere}
\begin{enumerate}

\item emission of photons: select two angles (3D space). In isotropic scattering
\begin{itemize}
\item $\theta$ met $\mu = \cos(\theta)$
	\begin{itemize}
	\item $\mu = 2\xi -1$ (isotropic scattering)
	\item $\mu = \sqrt{\xi}$ (A slab is heated from below. Then $P(\mu) = \mu$)
	\end{itemize}
\item $\phi = 2 \pi \xi$
\end{itemize}

\item propagation of photons
\begin{itemize}
\item sample optical depth from $\tau = -\log(\xi)$
\item distance travelled $L = \frac{\tau z_{max}}{\tau_{max}}$
\end{itemize}

\item conclusion of emission and propagation
\begin{equation}
\begin{aligned}
x &= x + L \sin(\theta) \cos(\phi) \\
y &= y + L \sin(\theta) \sin(\phi) \\
z &= z + L \cos(\theta)
\end{aligned}
\end{equation}

\item Binning: once the photon exists the slab. Produce histograms of the distribution function. Finally, we wish to compute the output flux or the intensity.
\end{enumerate}

I have seen that a newer version of the paper is available, which was also used in these notes (which contains amongst other up-to-date references to code fragments).

\paragraph{A Plane Parallel, Isotropic Scattering Monte Carlo Code}

\newpage
\subsection{Monte Carlo Radiative Transfer}
From a macroscopic perspective, RT calculations rest on the transfer equation
\begin{itemize}
\item emissivity $\eta$ (how much energy is added to radiation field due to emission)
\item opacity $\chi$ (how much energy is removed due to absorption)
\item the source function $S = \frac{\eta}{\chi}$
\item optical depth $\tau$ captures the opaqueness of a medium
\end{itemize}
\begin{equation}
\left( \frac{1}{c} \frac{\partial}{\partial t} + \nabla.n \right)I = \eta - \chi I
\end{equation}
\begin{equation}
d\epsilon = I d \nu dt d\Omega dA.n
\end{equation}

\newpage
\subsection{P Cygni profile for beta-velocity law and given opacity Monte Carlo simulation}
\subsubsection{Structure of the code}
\begin{itemize}
\item module common

\item module my\_inter

\item program pcyg
\begin{itemize}
\item INPUT xk0, alpha, beta
\item OUTPUT 
\item PROGRAM FLOW: loop over all photons
\begin{itemize}
\item get xstart and vstart
\item 
\end{itemize}
\item then do normalisation
\end{itemize}

\item function func(r)

\item function xmueout(xk0,alpha,r,v,sigma)

\item function rtbis(func,x1,x2,xacc)
\end{itemize}

\newpage
\section{Challenges in Radiative Transfer (Ivan Milic)}
\subsection{Overview of the problem}
\begin{center}
\begin{tikzpicture}
[node distance=2.5cm,auto,>=latex']
    \node [int] (a) {$T(\tau)$ , $\rho(\tau)$ , $\vec{B}(\tau)$ , $\vec{v}(\tau)$};
    \node (b) [left of=a,node distance=4cm, coordinate] {a};
    \node (c) [right of=a,node distance=4cm, coordinate] {a};
    \path[->] (b) edge node {$I_{\lambda}^*$} (a);
    \path[->] (a) edge node {$I_{\lambda}^+$} (c);
\end{tikzpicture}
\end{center}

\subsubsection*{Forward problem}
The forward problem is schematically represented
\begin{center}
\begin{tikzpicture}
[node distance=2.5cm,auto,>=latex']
    \node [int,align=center] (a) {forward problem \\
    $I_{\lambda}^+ = F(\vec{T},\rho,\vec{B},\vec{v})$};
    \node (b) [left of=a,node distance=4cm, coordinate] {a};
    \node (c) [right of=a,node distance=4cm, coordinate] {a};
    \path[->] (b) edge node {$\vec{T},\rho,\vec{B},\vec{v}$} (a);
    \path[->] (a) edge node {$I_{\lambda}^+$} (c);
\end{tikzpicture}
\end{center}
In fact solve for intensity vector $  \vec{I} = \left( \begin{matrix}
I \\ Q \\ \alpha \\ V
\end{matrix} \right) $
obeying the equation
\begin{equation}
\frac{d\vec{I}}{d\tau} = -X(\vec{T},\rho,\vec{B},\vec{v}) \vec{I} - \vec{j}(\vec{T},\rho,\vec{B},\vec{v})
\end{equation}
and the solution 
\begin{equation}
I_{\lambda}^+ = I_{0}^+ e^{- \int} + \int\vec{j}e^{-\int} d\tau
\end{equation}

\noindent\fbox{
  \parbox{\textwidth}{
\textbf{Example}
Source function $S=a\tau + b$ then $\int_0^{\tau_{max}} (a\tau+b)e^{-\tau} d\tau = ... $
}}

\subsubsection*{Inverse problem}
The inverse problem is schematically represented
\begin{center}
\begin{tikzpicture}
[node distance=2.5cm,auto,>=latex']
    \node [int,align=center] (a) {inverse problem \\ $F^{-1}(I_{\lambda}^{obs})$};
    \node (b) [left of=a,node distance=4cm, coordinate] {a};
    \node (c) [right of=a,node distance=4cm, coordinate] {a};
    \path[->] (b) edge node {$I_{\lambda}^+$} (a);
    \path[->] (a) edge node {$\vec{T},\rho,\vec{B},\vec{v}$} (c);
\end{tikzpicture}
\end{center}
Via least-squares approximation
\begin{equation}
\min_{\vec{T},\rho,\vec{B},\vec{v}} \sum \left( I_{\lambda}^{obs} - I_\lambda(\vec{T},\rho,\vec{B},\vec{v}) \right)^2
\end{equation}

\subsection{Challenging domains of application}
\begin{itemize}
\item Lyman alpha in Galaxy Halos
\item Dusty torii (AGD)
\item protoplanetary disks
\item circumstellar disks
\item athmospheres
\end{itemize}

\newpage
\section{Asymptotic Preserving Monte Carlo methods for transport equations in the diffusive limit (Dimarco+2018)}

\section{Fluid and hybrid Fluid-Kinetic models (for neutral particles in plasma edge) (Horsten2019)}
The material is mainly taken from \cite{HorstenNiels2019}.
\begin{itemize}
\item Kinetic Boltzmann equation: neutral velocity distribution $f_n(r,v)$
\item If you taken into account (e.g. microscopic processes for atomic deuterium) then the kinetic Boltzmann equation becomes 
\begin{equation}
v\nabla f_n(r,v) = S_r(r,v) + S_{cx}(r,v) - f_n(r,v)(R_{cx}(r,v)+R_i(r))
\label{Horsten_kinetic_equation}
\end{equation}

\item Numerical solution strategies
\begin{itemize}
\item finite differences/volumes/elements :computationally infeasible
\item spectral methods (series expansion of $f_n(r,v)$: not suitable for modelling discontinuties
\item stochatic approach: the whole velocity distribution is discretized by finite set of particles
\end{itemize}  

\item from Equation (\ref{Horsten_kinetic_equation}), the fluid model and the hybrid model is derived.
\begin{itemize}
\item Fluid model: 3 state equations (continuity -  momentum - energy) with boundary conditions
\begin{itemize}
\item pure-pressure equation: maximum error of 10 - 28 \% 
\item with parallel momentum source: error 10 \%
\item with ion energy source: error 30 \% 
\end{itemize} 

\item hybrid model based on micro-macro decomposition
\end{itemize}

\end{itemize}


\newpage
\section{Splitting methods}
From notes by professor Frank.
\subsection{Exercises}
\subsubsection{Exercise 1}

\newpage
\section{Overview of existing (Monte Carlo) radiative transfer codes}

\subsection{Synthesis codes}
As is pointed out in \cite{PinteChristophe2015CRT}, there are basically two methods to solve the radiative transfer problem: ray-tracing and Monte Carlo methods.

\begin{itemize}
\item RADICAL \cite{RADICAL}  (Ray-tracing, 2D, multi-purpose)
\item MULTI \cite{MULTI} \cite{Carlsson1986} (computer program for solving multi-level non-LTE radiative transfer problems in moving or static atmospheres, very old: Uppsala 1986 - 1995)
\item SKIRT \cite{Camps2015} (continuum (Monte Carlo) radiation transfer in dusty astrophysical systems, such as spiral galaxies and accretion disks, from Ugent)
\item TORUS \cite{Harries2019} (Monte Carlo radiation transfer and hydrodynamics code. Adopts 1D, 2D, 3D adaptive mesh refinement. Suitable for radiative equilibrium and creation of synthetic images and SED)
\item RADMC-3D \cite{RADMC3D} (Monte Carlo code that is especially applicable for dusty molecular clouds, protoplanetary disks, circumstellar envelopes, dusty tori around AGN and models of galaxies. Python interface with Fortran main code)
\item TLUSTY and SYNSPEC \cite{Hubeny2017}, \cite{Hubeny2017a}, \cite{Hubeny2017b}.
\end{itemize}

\subsection{Inversion codes}
\begin{itemize}
\item VFISV
\item ASP/HAO
\item HeLIx+
\item SNAPI (not publicly available, created by Ivan Milic)
\item multiple codes available from Instituto de Astrofyiica de Canarias (IAC)
\item STiC: the Stockholm inve rsion code
\end{itemize}

\newpage
\section{Integral equations}
Based on the book \cite{Mmfp}.
\begin{enumerate}
\item integral equation from differential equation
\item types of integral equations
\item operator notation and existence of solutions
\item closed-form solutions
\begin{itemize}
\item separable kernels
\item integral transform method (Fourier transform)
\item differentiation
\end{itemize}

\item Neumann series
\item Fredholm theory
\item Schmidt-Hilbert theory
\end{enumerate}

Fredholm equation first kind
\begin{equation}
0 = f + \lambda \mathcal{K}y
\end{equation}

Fredholm equation second kind
\begin{equation}
y = f + \lambda \mathcal{K}y
\end{equation}


\end{document}
