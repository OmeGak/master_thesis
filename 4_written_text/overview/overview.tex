\documentclass[../main/main.tex]{subfiles}
\begin{document}

\section{Very broad introduction \& Summary}
The material here originates from the master thesis of Nicolas Moens \cite{MoensNicolas} and from the course notes \textit{Introduction to numerical methods for radiation in astrophsyics} from professor Sundqvist. 

\subsection{Definition of specific intensity}
\label{specific_intensity}
The definition of the specific intensity is 
\begin{equation}
I_{\nu} 
	= \frac{dE_{\nu}}{\cos(\theta) d\Omega dt d\nu} 
	= \frac{dE_{\nu}}{\mu d\Omega dt d\nu} 
\end{equation}
On the other hand, for the total energy of a collection of $N$ photons holds that 
\begin{equation}
E_{\nu} = N E_{\nu,\text{photon}} 
\end{equation}

\paragraph{To the point}
From this we deduce that 
\begin{equation}
I_{\nu} \mu  = \frac{N(\mu) dE_{\nu,\text{photon}}}{d\Omega dt d\nu}
\end{equation}
and thus 
\begin{equation}
\boxed{I_{nu} \mu d\mu \sim N(\mu) d\mu}
\end{equation}

\paragraph{Considering the solid angle}
In spherical geometry $d\Omega = \sin(\theta) d\theta d\phi = d\mu d\phi$.

\subsection{Radiation equations}
Material from \cite{TheoryStellarAtmospheres2014}

Specific intensity $I(s,\lambda,x,y,t)$ 
\begin{equation}
\frac{\delta I(q,t)}{\delta s} = \eta (q,t) - \chi(q,t) I(q,t)
\end{equation}
In cartesian coordinates (with propagation vector $\vec{n} = \left[ \begin{matrix} n_x \\ n_y \\ n_z \end{matrix}  \right]  = \left[ \begin{matrix} \sin(\theta) \cos(\phi) \\ \sin(\theta) \sin(\phi)  \\ \cos(\theta) \end{matrix}  \right]$):
\begin{equation}
\frac{1}{c}\frac{\partial I}{\partial t} + \sin(\theta)\cos(\phi)\frac{\partial I}{\partial x} + \sin(\theta)\sin(\phi)\frac{\partial I}{\partial y} + \cos(\theta) \frac{\partial I}{\partial z} = \eta - \chi I
\end{equation}
\begin{itemize}
\item 1D planar atmosphere: $\frac{\partial I}{\partial x} = \frac{\partial I}{\partial y} = 0$:
\begin{equation}
\frac{1}{c} \frac{\partial I}{\partial t} + \mu \frac{\partial I}{\partial z} = \eta - \chi I
\end{equation}

\item diffusion limit
\item Definition of $J$ in Equation (3.15)
\end{itemize}

\paragraph{Plane parallel geometry}
\begin{itemize}
\item restrict oursevels to time-independent, one-dimensional (1D) case $I(s,\theta,\lambda)$ where $s$ is the direction of the light ray
%\item defining $I_{\lambda} = \frac{I}{-\chi_{\lambda}}$ 
\item it satisfies Radiation Transfer Equation (RTE) 
$\boxed{\frac{dI_{\lambda}}{d\tau_{\lambda}} = S_{\lambda} - I_{\lambda}}$
\item with 'formal' solution 
$\boxed{I(\lambda,\tau_{\lambda}) = I_0(\lambda) e^{-\tau_{\lambda}}  \int_0^{\tau_{\lambda}}S(t)e^{-t} dt}$
\begin{itemize}
\item no emissivity $S=0$ then $I(\lambda) I_0(\lambda) e^{-\tau_{\lambda}}$
\item no opacity then $I_0(\lambda) = \int_0^{s} \eta_{\lambda}(s)ds$ 
\item constant source function $I(\lambda,\tau) = I_0(\lambda) e^{-\tau_{\lambda}} + S(1-e^{-\tau_{\lambda}})$
\item if $S=a+b\tau$ then $I(\lambda) = a+\frac{b}{k_{\lambda}}$ with $k_{\lambda}$ the opacity. A jump in opacity leads to the jump in intensity of the opposite sign.
\end{itemize}
\end{itemize}

\paragraph{Specific intensity and its angular moments}

\begin{center}
\centering
{\tabulinesep=1.5mm
\begin{tabu}{|c|c|}
\hline 
specific intensity & $\Delta \epsilon = \boxed{I_{\nu}} A_1 A_2/r^2 \Delta \nu \Delta t$ \\ \hline
energy density & $E = \frac{1}{c} \iint I_{\nu} d\nu d\Omega$ \\ \hline
flux vector & $F = \iint I_{\nu}n d\nu d\Omega$ \\ \hline
pressure tensor & $P = \iint I_{\nu} nn d\nu d\Omega$ \\ \hline
mean intensity & $J_{\nu} = \frac{c}{4 \pi} E_{\nu}$ \\ \hline
Eddington flux & $H_{\nu} = \frac{1}{4 \pi} F_{\nu}$ \\ \hline
Eddington's K & $K_{\nu} = \frac{c}{4 \pi}P_{\nu}$ \\ \hline
\end{tabu}}
\end{center}

\paragraph{Eddington factor}
In general, the Eddington factor is a tensor, for 1D systems it is reduced to a scalar.
\begin{equation}
f_{\nu} = \frac{K_{\nu}}{J_{\nu}} = \frac{P_{\nu}}{E_{\nu}}
\end{equation}
\begin{itemize}
\item isotropic radiation field
\item radiation field stronly peaked in radial (i.e. vertical in cartesian) direction
\end{itemize}



\subsection{Radiative Diffusion Approximation} The radiative diffusion approximation bridges two regimes: regimes with ... 
\begin{itemize}
\item on one hand, large optical depth $\tau \gg 1$: diffusion equation: temperature structure in a static stellar atmosphere
\item on the other hand, where radiative \textit{transport} is important
\end{itemize}
The diffusive approximation is the following: replace $\boxed{I = B}$ or $I_{\nu} = B_{\nu}$.
\begin{equation}
I_{\nu} = B_{\nu} - \mu \frac{dB_{\nu}}{k_{\nu}dz}
\end{equation}
This equation can be derived as a random walk of photons!


\subsection{Applications and approximations for radiative forces}
\begin{itemize}
\item definition of general radiative acceleration vector $g_{\text{rad}} = \frac{1}{\rho c}\int \int n k_{\nu} I_{\nu} d\Omega d\nu$
\end{itemize}

\subsection{RHD equations}
The full RHD equations consist of 
\begin{itemize}
\item five partial differential equations
\item one HD closure equation, e.g. (i) variable Eddington tensor method or (ii) flux limited diffusion
\end{itemize}

\paragraph{Heat flux}
The heat flow rate density $\vec{\phi}$ satisfies the Fourier law $\vec{\phi} = - k\nabla T$. More information can be found for instance on \cite{WikiHeat}.

\subsection{Overview of symmetry assumptions}
\begin{center}
\centering
{\tabulinesep=1.5mm
\begin{tabu}{|c|c|c|}
\hline 
plane-parallel & 1D atmosphere & \\ 
& bounded by horizontal surfaces & \\ \hline
\end{tabu}}
\end{center}


\subsection{Overview of units}
\begin{center}
\centering
{\tabulinesep=1.5mm
\begin{tabu}{|c|c|}
\hline 
opacity $\alpha = k_{\nu}$ & $\left[ \frac{m^2}{kg} \right] $ \\ \hline

specific intensity $I_{\nu}$ & $\left[ \frac{ergs}{cm^2 . sr . Hz . s} \right] = \left[ \frac{J}{cm^2 . sr . Hz . s} \right] $ \\ \hline

optical depth $\tau$ & \\ 
& $\boxed{\tau = 0}$ leave atmosphere \\ \hline
\end{tabu}}
\end{center}

\subsubsection{Things to know}
\begin{itemize}
\item expanding flow: redshift (lower frequency)
\item compressing flow: blueshift (higher frequency)
\end{itemize}


\newpage
\section{The mathematics of Radiative Transfer}
The material in this section is based on the book \cite{Busbridge}.

\subsection{Auxiliary mathematics}

\begin{itemize}
\item $\cos(\Theta) = cos(\theta)\cos(\theta') + \sin(\theta)\sin(\theta') \cos(\phi-\phi')$

\item phase function $\boxed{p(\mu,\phi,\mu',\phi',\tau) = \sum_{n=0}^N \omega_n P_n(\cos(\Theta))}$
\begin{itemize}

\item isotropic scattering $p(\tau) = \omega_0(\tau)$
\end{itemize}

\item equation of transfer $\boxed{\mu \frac{\partial I(\tau,\mu,\phi)}{\partial \tau} = I(\tau,\mu,\phi) - \mathcal{S}(\tau,\mu,\phi)}$ 
\\ with $\mathcal{S}(\tau,\mu,\phi) = B_1(\tau) + \frac{1}{4 \pi} \int_{-1}^{1} d\mu' \int_0^{2\pi} I(\tau,\mu',\phi') p(\mu,\phi,\mu',\phi') d\phi'$
\begin{itemize}
\item axially symmetric with isotropic scattering \\
$\mathcal{S}(\tau) = \frac{\omega_0(\tau)}{2} \int_{-1}^{1} I(\tau,\mu') d\mu' = 
B_1(\tau) + \frac{\omega_0(\tau)}{2} \int_0^{\tau_1} \mathcal{S} (t) E_1(|t-\tau|)dt $
\item the Milne equation of the problem $(1-\omega_0 \bar{\Lambda})\{ \\mahtcal{S}(t)\} = B(\tau)$
\begin{itemize}
\item solve for $\mathcal{S}(t)$
\item then find $I(\tau,\mu)$
\end{itemize}

\end{itemize}
\end{itemize}

\subsection{The H-functions}
\begin{itemize}
\item characteristic equation
\end{itemize}

\subsection{Integral equations}
Based on the book \cite{Mmfp}.
\begin{enumerate}
\item integral equation from differential equation
\item types of integral equations
\item operator notation and existence of solutions
\item closed-form solutions
\begin{itemize}
\item separable kernels
\item integral transform method (Fourier transform)
\item differentiation
\end{itemize}

\item Neumann series
\item Fredholm theory
\item Schmidt-Hilbert theory
\end{enumerate}

Fredholm equation first kind
\begin{equation}
0 = f + \lambda \mathcal{K}y
\end{equation}

Fredholm equation second kind
\begin{equation}
y = f + \lambda \mathcal{K}y
\end{equation}

\newpage
\section{Challenges in Radiative Transfer}
The material here originates from an oral discussion with Ivan Milic. 
\subsection{Overview of the problem}
\begin{center}
\begin{tikzpicture}
[node distance=2.5cm,auto,>=latex']
    \node [int] (a) {$T(\tau)$ , $\rho(\tau)$ , $\vec{B}(\tau)$ , $\vec{v}(\tau)$};
    \node (b) [left of=a,node distance=4cm, coordinate] {a};
    \node (c) [right of=a,node distance=4cm, coordinate] {a};
    \path[->] (b) edge node {$I_{\lambda}^*$} (a);
    \path[->] (a) edge node {$I_{\lambda}^+$} (c);
\end{tikzpicture}
\end{center}

\subsubsection*{Forward problem}
The forward problem is schematically represented
\begin{center}
\begin{tikzpicture}
[node distance=2.5cm,auto,>=latex']
    \node [int,align=center] (a) {forward problem \\
    $I_{\lambda}^+ = F(\vec{T},\rho,\vec{B},\vec{v})$};
    \node (b) [left of=a,node distance=4cm, coordinate] {a};
    \node (c) [right of=a,node distance=4cm, coordinate] {a};
    \path[->] (b) edge node {$\vec{T},\rho,\vec{B},\vec{v}$} (a);
    \path[->] (a) edge node {$I_{\lambda}^+$} (c);
\end{tikzpicture}
\end{center}
In fact solve for intensity vector $  \vec{I} = \left( \begin{matrix}
I \\ Q \\ \alpha \\ V
\end{matrix} \right) $
obeying the equation
\begin{equation}
\frac{d\vec{I}}{d\tau} = -X(\vec{T},\rho,\vec{B},\vec{v}) \vec{I} - \vec{j}(\vec{T},\rho,\vec{B},\vec{v})
\end{equation}
and the solution 
\begin{equation}
I_{\lambda}^+ = I_{0}^+ e^{- \int} + \int\vec{j}e^{-\int} d\tau
\end{equation}

\noindent\fbox{
  \parbox{\textwidth}{
\textbf{Example}
Source function $S=a\tau + b$ then $\int_0^{\tau_{max}} (a\tau+b)e^{-\tau} d\tau = ... $
}}

\subsubsection*{Inverse problem}
The inverse problem is schematically represented
\begin{center}
\begin{tikzpicture}
[node distance=2.5cm,auto,>=latex']
    \node [int,align=center] (a) {inverse problem \\ $F^{-1}(I_{\lambda}^{obs})$};
    \node (b) [left of=a,node distance=4cm, coordinate] {a};
    \node (c) [right of=a,node distance=4cm, coordinate] {a};
    \path[->] (b) edge node {$I_{\lambda}^+$} (a);
    \path[->] (a) edge node {$\vec{T},\rho,\vec{B},\vec{v}$} (c);
\end{tikzpicture}
\end{center}
Via least-squares approximation
\begin{equation}
\min_{\vec{T},\rho,\vec{B},\vec{v}} \sum \left( I_{\lambda}^{obs} - I_\lambda(\vec{T},\rho,\vec{B},\vec{v}) \right)^2
\end{equation}

\subsection{Challenging domains of application}
\begin{itemize}
\item Lyman alpha in Galaxy Halos
\item Dusty torii (AGD)
\item protoplanetary disks
\item circumstellar disks
\item athmospheres
\end{itemize}


\newpage
\section{Glossary}
\begin{itemize}
\item SED: \hfill spectral energy distribution

\item (spectral) line-force: \hfill force on material in stellar atmosphere

\item LASER: \hfill Light Amplification by Stimulated Emission of Radiation
\end{itemize}

\end{document}
