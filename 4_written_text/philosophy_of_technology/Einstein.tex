\documentclass[../main/main.tex]{subfiles}

\begin{document}


\newpage
\section{Iets anders, Einstein}
Een centrale vraag die ik me stel is in feite de volgende: is Einstein (attributief) verantwoordelijk voor de dood van mensen die zijn omgekomen door de werking van de atoombom?
\begin{itemize}
\item Indien ja, vind ik het moeilijk om te zeggen dat er een responsability gap ontstaat bij killer robots. Immers de ontwerper van de robot moet zijn product maar op die manier programmeren dat de ontwerper kan voorspellen wat de robot gaat doen, voor alle mogelijke omstandigheden waarin de robot kan terechtkomen. Indien de ontwerper dat niet kan, is hij onvoldoende geschikt voor het uitvoeren van zijn taak. Ik zeg dus dat de ontwerper slechts datgene mag ontwerpen waarvan hij de gevolgen volledig kan overzien, hoe radicaal ook.
\item Indien neen is het probleem een stuk complexer.
\end{itemize}


Echter, in mijn ogen lijkt de tweede optie (antwoord neen) niet aaneemelijk. Stel dat Einstein geen verantwoordelijkheid draagt, dan kan men ook zeggen dat ... er dingen kunnen gebeuren die duidelijk door mensen in gang zijn gezet, maar waar niemand met (attributieve) verantwoordelijkheid aansprakelijk voor is. In dat geval gaat er volgens mij verantwoordelijkheid verloren in de wereld (bestaan een soort van 'verantwoordelijkheids-zinkgat'). Een vrij mens kan doen wat hij wil, maar draagt daar ook verantwoordelijkeid voor - misschien wel tot het extreme toe. Natuurlijk moet men hier een onderscheid maken tussen causale en attributieve verantwoordelijkheid; en de vraag is dan natuurlijk waar precies de grens ligt. 

\end{document}