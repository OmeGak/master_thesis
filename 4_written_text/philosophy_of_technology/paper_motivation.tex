\documentclass[../main/main.tex]{subfiles}

\begin{document}
\section{[Philosophy of Technology] - Motivation for paper \\ Ignace Bossuyt}

I am very interested in the concept of (attributive) responsibility and its importance in society and in the develpment of technology. At one hand, I would like to discuss the (necessary) conditions for attributive responsibility, its place amongst other forms of responsibility and its relation with culpability.

More specically, I will study the difficulties that arise when multiple actors are involved into some harm (the so-called problem of many hands). The presented paper [1] applies this problem to the problem of climate change, an ongoing problematic with has gained large media coverage recently. This problem is relevant for me personally both as a civilian (what about my responsibility for climate change as consumer/liver) and as a (future) engineer (what about responsibility for climat change as a designer).

My paper would also be a follow-up of the discussion that I had with the professor on 28/10, for which I would also like to thank him very much.

\begin{thebibliography}{12}  
\bibitem{Problem_many_hands_artikel} Ibo van de Poel, Jessica Nihlén Fahlquist, Neelke Doorn, Sjoerd Zwart, Lamber Royakkers, ``The Problem of Many Hands: Climate Change as an Example", July 5, 2010, Springer.

\end{thebibliography}

\end{document}