\documentclass[../main/main.tex]{subfiles}

\begin{document}

\part{E-mail with professor}

\section{Thought on Responsability}

\subsection{Observation 1}
For some deterministic systems (more specifically \textit{chaotic systems}, a small change in initial condition can alter the long-term behaviour of the system very strongly. In the inverse sense, it is very diffucult to determine correctly the initial condition from a given long-term behaviour. This observation is the basis for some system in cryptography (i.e. easy in one way, very difficult for an attacker without the knowledge of how encryption worked in the other way).

\subsection{Observations 1 applied to causal responsability}
\label{observation_causal_respo}
In a causal sense, there is always someone responsible. However, based upon Observation 1, correctly tracing back who exactly is causally responsible for certain actions can be very hard. For example, who is responsible for the studies that student X undertakes with great honours (his/her parents, school teacher...). One may then assign partial responsabilities. 


\subsection{Open questions}
The questions that are bothering my mind are the following:
\begin{enumerate}
\item What's the precise difference between attributive and causal responsability? Maybe causal responsability does not lead to attributive responsability in cases where an action happened by chance (without intention) or by lack of knowledge. 

\item Does the reasoning from Section \ref{observation_causal_respo} also hold for attributive responsability? This question is also concerned with the question to what extent attributive responsability is necessary/wishable in every-day life. Here one example comes to my mind. After the 2019 fire in Paris' Notre Dame, people filed a complaint to unknowns (news article [1]). This shows that, in some sense, responsability is expected in society. It is very hard to trace back who could be held responsible for 'not warning for lead intoxication'.

\item Is Einstein responsible for the death of people? Maybe he could have known that people can use his inventions maleficiently? More generally, this question addresses the ethics of (pure) sciences. 

This question also deals with the definition of the frontier (if there even is a frontier) of a designer's responsability when he designs a system. One could perhaps argue that a designer who cannot predict the behaviour of his designed product in all possible circumstances, should not be allowed to produce that product, as he is then unable to control his product's actions. The production of such a product is then reprehensible for the same reason why it is reprehensible to loose a lion on the streets (one is not completely sure if the lion will act agressively or not). Here enters a possible connection with killer robots. 
\end{enumerate}

\begin{thebibliography}{100}  
\bibitem{VRT_artikel} Bert De Vroey, ``Klacht tegen onbekenden na brand in Notre-Dame: "Men had moeten waarschuwen voor loodvergiftiging"" \emph{VRT Nieuws}, July 30, 2019, \url{https://www.vrt.be/vrtnws/nl/2019/07/30/klacht-tegen-onbekenden-na-brand-in-notre-dame-de-overheid-had/}.
\end{thebibliography}


\part{Einstein}
\section{Iets anders, Einstein}
Een centrale vraag die ik me stel is in feite de volgende: is Einstein (attributief) verantwoordelijk voor de dood van mensen die zijn omgekomen door de werking van de atoombom?
\begin{itemize}
\item Indien ja, vind ik het moeilijk om te zeggen dat er een responsability gap ontstaat bij killer robots. Immers de ontwerper van de robot moet zijn product maar op die manier programmeren dat de ontwerper kan voorspellen wat de robot gaat doen, voor alle mogelijke omstandigheden waarin de robot kan terechtkomen. Indien de ontwerper dat niet kan, is hij onvoldoende geschikt voor het uitvoeren van zijn taak. Ik zeg dus dat de ontwerper slechts datgene mag ontwerpen waarvan hij de gevolgen volledig kan overzien, hoe radicaal ook.
\item Indien neen is het probleem een stuk complexer.
\end{itemize}


Echter, in mijn ogen lijkt de tweede optie (antwoord neen) niet aaneemelijk. Stel dat Einstein geen verantwoordelijkheid draagt, dan kan men ook zeggen dat ... er dingen kunnen gebeuren die duidelijk door mensen in gang zijn gezet, maar waar niemand met (attributieve) verantwoordelijkheid aansprakelijk voor is. In dat geval gaat er volgens mij verantwoordelijkheid verloren in de wereld (bestaan een soort van 'verantwoordelijkheids-zinkgat'). Een vrij mens kan doen wat hij wil, maar draagt daar ook verantwoordelijkeid voor - misschien wel tot het extreme toe. Natuurlijk moet men hier een onderscheid maken tussen causale en attributieve verantwoordelijkheid; en de vraag is dan natuurlijk waar precies de grens ligt. 

\newpage
\part{Paper}
I am very interested in the concept of attributive responsability and its importance in society. On one hand, I would like to look at its (theoretical) foundations by discussing the (necessary) conditions for attributive responsability and its place amongst other forms of responsability. On the other hand, I will look into real-life problems involving responsability to show the relevance of responsability in society. 

More specically, I will study the difficulties that arise when multiple actors are involved into some harm (the so-called problem of many hands). The presented paper [1] applies this problem to the problem of climate change, an ongoing problematic with has gained large media coverage recently. 

This is also a follow-up of the discussion that I had with the professor, for which I would also like to thank him very much.

\vfill
\begin{thebibliography}{12}  
\bibitem{Problem_many_hands_artikel} Ibo van de Poel, Jessica Nihlén Fahlquist, Neelke Doorn, Sjoerd Zwart, Lamber Royakkers, ``The Problem of Many Hands: Climate Change as an Example", July 5, 2010, Springer.

\end{thebibliography}

\end{document}