\documentclass[../main/main.tex]{subfiles}

\begin{document}

\section{Thought on Responsability}

\subsection{Observation 1}
For some deterministic systems (more specifically \textit{chaotic systems}, a small change in initial condition can alter the long-term behaviour of the system very strongly. In the inverse sense, it is very diffucult to determine correctly the initial condition from a given long-term behaviour. This observation is the basis for some system in cryptography (i.e. easy in one way, very difficult for an attacker without the knowledge of how encryption worked in the other way).

\subsection{Observations 1 applied to causal responsability}
\label{observation_causal_respo}
In a causal sense, there is always someone responsible. However, based upon Observation 1, correctly tracing back who exactly is causally responsible for certain actions can be very hard. For example, who is responsible for the studies that student X undertakes with great honours (his/her parents, school teacher...). One may then assign partial responsabilities. 


\subsection{Open questions}
The questions that are bothering my mind are the following:
\begin{enumerate}
\item What's the precise difference between attributive and causal responsability? Maybe causal responsability does not lead to attributive responsability in cases where an action happened by chance (without intention) or by lack of knowledge. 

\item Does the reasoning from Section \ref{observation_causal_respo} also hold for attributive responsability? This question is also concerned with the question to what extent attributive responsability is necessary/wishable in every-day life. Here one example comes to my mind. After the 2019 fire in Paris' Notre Dame, people filed a complaint to unknowns (news article [1]). This shows that, in some sense, responsability is expected in society. It is very hard to trace back who could be held responsible for 'not warning for lead intoxication'.

\item Is Einstein responsible for the death of people? Maybe he could have known that people can use his inventions maleficiently? More generally, this question addresses the ethics of (pure) sciences.
\end{enumerate}


\vfill
\begin{thebibliography}{100}  
\bibitem{VRT_artikel} Bert De Vroey, ``Klacht tegen onbekenden na brand in Notre-Dame: "Men had moeten waarschuwen voor loodvergiftiging"" \emph{VRT Nieuws}, July 30, 2019, \url{https://www.vrt.be/vrtnws/nl/2019/07/30/klacht-tegen-onbekenden-na-brand-in-notre-dame-de-overheid-had/}.
\end{thebibliography}

\end{document}