\documentclass[../main/main.tex]{subfiles}

\begin{document}

\part{Topic 1: Enchancement}
\section{Lecture}
\paragraph{Objectives of technology}
\begin{itemize}
\item cure
\item prevent
\item enhance
\begin{itemize}
\item strenghten what exists
\item introduce new
\end{itemize}
\end{itemize}

\paragraph{Anti-enhancement (conservatifs)}
Habermas, Sandel, Fukuyama

\paragraph{Liberals: anti anti-enhancement}

\paragraph{Critisizing assumptions}

\paragraph{Critisizing Arguments}

\newpage
\section{The case against perfection: what's wrong with designer children, bionic athletes, genetic engineering}
Michael J. Sandel.

The text is divided in the following subsections
\begin{enumerate}
\item Muscles \hfill athletes (muscle enhancement)
\item Memory \hfill (cognitive enhancement)
\item Heigth \hfill (growth-hormone treatment)
\item Sex selection \hfill reproductive technologies (sex and some genetic traits)
\end{enumerate}

\newpage
\section{Debate}
\subsection{Question 1} 
\begin{itemize}
\item We have strong evidence, that our biological species-normal state is not 100\% fitted to the industrialized modern world. Obesity, higher cancer rates, allergies, ADHD, anxiety and depression statistics shows that \textbf{humanity is ill-equipped to the challenges of the current state of the world}. Some companies and actors even using humanities evolutionary instincts against itself. (Market predication, HFCS, Psychological Operation, attention economy)

\item To combat the evolutionary unpreparedness to the modern condition can be viewed as cure and not even an enhancement, but nonetheless \textbf{it is desirable, and morally acceptable to enhance humanity, and raising the species-normal state, as it would be an effective way to combat the vulnerability of our species against the maleficent manipulation of our biological preparedness}. 

\item Do you agree with this statement? If not, what are your counterarguments? 

\begin{itemize}
\item My response to this statement rests on a holistic approach.

\item The current state of the world is of course a product of human mankind and intellect. It is actually the product of technology. Defeating tehchnology with technology is absurd (if you want, compare with Cold War). Thus, it is not morally acceptable to enhance humanity because of the reason in the statement. The problem in the statement should be solved throug reducing the development of the technology that creates this \textit{maleficient manipulation of our biological preparedness}.

\item The sentence \textit{humanity is ill-equiped to the challenges of the current state of the world} is in the wrong direction. It should be \texttt{the current state of the world is ill-equiped to the true nature of mankind}.
\end{itemize}

\end{itemize}


\subsection{Question 2}  Do you agree with Michael J. Sandel, on that the solidarity of society is greatly depend on the genetic lottery and the uneven, and unpredictable distribution of talent?

\begin{itemize}
\item There is indeed a \textit{stochastic} distribution of talent (fitness). One way to see this, is as a gift from God (see question 4). The \textit{genetic lottery} and the \textit{uneven, unpredictable distribution of talent} indeed contribute to solidarity, but are not the only reasons.

\item To what extend is the \textit{web} where you are born determined by your genes?

\item What are the roots of solidarity? 
\begin{itemize}
\item etymology: Latin \textit{solidus} (solid)
\item solidarity is a means to render society more solid
\item Latin \textit{socius}: partner, allied
\end{itemize}
\end{itemize}

\subsection{Question 3} How could genetic enhancement of the kind discussed in the text disrupt what it means to be human?
\begin{itemize}

\item Let us first discuss what it means to be human. 
\begin{itemize}
\item humanity, from Latin \textit{humanitas} which means (i) human culture, (ii) kindness, courtesey,(iii) culture, civilization
\item We could look at human rights, but one very important thing is freedom.
\begin{itemize}
\item All human beings are born free and equal in dignity and rights. They are endowed with reason and conscience and should act towards one another in a spirit of brotherhood.
\item  ...
\item Everyone has the right to life, liberty and security of person.
\end{itemize}
\item We have the \texttt{what} and the \texttt{why} (intention/motivation). 
\end{itemize}

\item Assume that genetic enhancement of the kind discussed in the text disrupts what it means to be human. \textsc{But a counterargument is that} ... of course, eventually, all \textit{random} factors are taken away. But then you could say that even that is human, because even genetic enhancement is created by humans.

\item Is perfection in contrast to what it means to be human? Prometheus: hubris?


\end{itemize}

\subsection{Question 4} Without the assumption of a god and of any religion, on what grounds is the 'giftedness of life' reasoning based?
\begin{itemize}
\item \textsc{Background of this question}: for religious persons (persons who have a faith), it is not difficult to understand the reasoning whithin faith. However, for outsiders, it is not easy for them to understand that because understaning faith require in fact \textit{a priori} faith (kind of mystic reasoning). 
\begin{itemize}
\item For example, how can God accept bad things such as diseases?
\begin{itemize}
\item not scientific story:
\item Adam and Eve ... They wanted to \textit{be} God, without his grace.
\end{itemize}
\item see also p. 78
\end{itemize}

\item \textsc{About discernment between cure and enhance} From theology, the discernment between healing and enhancement is problematic in the same way as it is in general. In some sense, no strict moral objection can be made against genetic enhancement, \textsc{BUT it is very important to take into account the \texttt{intention}!} In Greece e.g., the practice of \texttt{ekonomia} (economy, implying good or prudent handling) (in contrast to leaglsm - strict adherence to the letter of the law of the church).

\item If you are not assuming any god... a baby does not choose itself to be born. It \textit{gets} life from its parents/ from nature... BUT the thing is, \textsc{after a few generation of genetically altered parents, can we still consider a child as a gift from nature?}

\item human organs are not designed by humans. We have them and we can use them. But could we alter them?
\end{itemize}
 
 
\newpage
\section{Glossary}
\begin{itemize}
\item quandary \hfill dilemma, lastig parket
\item humility \hfill nederigheid
\end{itemize}


\newpage
\part{Topic 2: Killer Robots}

\section{Random Thought}
\paragraph{Observation}
For a deterministic system, a small change in initial condition can alter the long-term behaviour of a given system very strongly. In the inverse sense, it is very diffucult to determine the initial condition from a given long-term behaviour correctly. This observation is also the basis for some system in cryptography (i.e. easy in one way, very difficult in the other way without the knowledge of how encryption worked in the other way).

\paragraph{Observation 2}
In a causal sense, there is always a responsible. 

\paragraph{Application to concept of responsability}
\begin{itemize}
\item I claim that for \texttt{causal responsability}, one player with a very small contribution can have a very significant implact on what happens in a given society. As a (silly) example, I can claim that if the Apple that Newton saw falling (a story which might even not be correct haha) then maybe he had not discovered the Newton's law. In a certain sense, he is responsible for the fact that yesterday people where killed during a rocket attack (the rocket was launched with the help of calculation using Newton's law).

\item I agree that causal responsability does not coincide with morel responsability (e.g. the example from class that a scientist lets fall a recipient with a dangerous liquid on the ground and as a result some of his collegues are killed). I am not going to talk about the discernment between these types of responsability.

\item I claim here however, based upon the observation above, that correctly rooting back who is responsible for certain actions can be very,very tidious. We may assign partial responsabilies.

\item This then also comes down to questions like \textit{was Einstein responsible for the death of people?}.
\end{itemize}

\paragraph{Random thing}
VRT: klacht tegen onbekenden


\paragraph{Random thing}
It this actually goes all about the definition of \texttt{atrirubute} responsability: the reason for doing a thing.

\paragraph{Thinking of examples where causal responsability does NOT lead to attributive responsability}
\begin{itemize}
\item dat flesje vergif laten vallen op de grond
\end{itemize}
Ik denk dat al deze gevallen kunnen ondergebracht worden in volgende categorien:
\begin{itemize}
\item per ongeluk (error)
\item niet weten
\end{itemize}

Het ding is, ik weet niet goed of verantwoordelijkheid echt nodig is, het zou er toe leiden dat Einstein verantwoordelijk is voor de door van mensen.

\subsubsection{Conclusion}
Het komt er op neer dat ik het niet volledig met u eens ben, want dan zou je ook kunnen zeggen dat Einstein verantwoordelijk is voor de dood van mensen.

I think that responsability is important because otherwise how can you explain any maleficient action? The natural world as we know it does not have maleficient actions.


\newpage
\section{Woordenlijst}
\begin{itemize}
\item compliance \hfill overeenstemming
\item contingent \hfill voorwaardelijk
\item lethal	\hfill dodelijk
\item LAWS	\hfill (Lethal Autonomous Weapons) - synonym for \textit{killer robot}

\item transgress \hfill overtreden
\item categorical \hfill indeling in categoriën

\end{itemize}

\newpage
\part{Research, philosofical paper}
\begin{itemize}
\item eugenics is bad, I agree.
\begin{itemize}
\item but the fact that is enhancement, I disagree.	
\item if we want to know that we should study long-term effects
\end{itemize}
\end{itemize}

\newpage
\part{Extra material}
In the light of (full) space and time, it makes no sense to take a certain standpoint - because it might always change. In addition, I do not pretend to have answers to any question. We can only make \textit{for the time being} partial answers. 

\paragraph{Iets waar ik niet tegen kan} mensen die tegen een computer praten.

Pas Wittgenstein to op phil of tech

\end{document}