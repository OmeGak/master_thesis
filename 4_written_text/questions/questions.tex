\documentclass[../main/main.tex]{subfiles}

\begin{document}
\section{Questions for professor Sundqvist}
\begin{itemize}
\item book \textit{Stellar Atmospheres} [Mihalas].
\item (for which star are the exerpimental data and what assumptions are used in the theory?)
\end{itemize}

\newpage
\section{Questions for professor Samaey}
\begin{itemize}
\item what is the difference between Monte Carlo and equation-free computing?
\end{itemize}

\newpage
\section{Solved questions}
\begin{itemize}
\item Sundqvist+ 2009: what is thermal velocity (see Wikipedia)
\item Sundqvist+ 2009: what is line force (see explanation Dylan)
\item unclassified: what is a flux limiter? (see course notes)
\item unclassified: what is cross section of scattering (see Wikipedia)
\item Puls manual: p.26: how does the Milne equation appear? (see library book)

\item pcyg.f90: what are p-rays? (see anwser professor Sundqvist)
\begin{itemize}
\item parallel rays leaving the atmosphere (of, e.g. a star)
\begin{center}
\begin{tikzpicture}
\begin{scope}
	\def \radij{2}	
	\def \dist{3}
  	\draw (0,0) circle (\radij);
	\draw[->] (\radij,0) -- (\radij+\dist,0);
	\draw[->] ({\radij*cos(30)},{\radij*sin(30)}) -- ({\radij*cos(30)+\dist},{\radij*sin(30)});
	\draw[->] ({\radij*cos(-30)},{\radij*sin(-30)}) -- ({\radij*cos(-30)+\dist},{\radij*sin(-30)});
\end{scope}
\end{tikzpicture}
\end{center}
\end{itemize}  

\item pcyg.f90: what is meant by Eddington limb-darkening? (see answer professor Sundqvist)
\begin{itemize}
\item standard limb darkening
\end{itemize}
\item Sundqvist+ 2009: what is the geometry of a \textit{slice}?

\item CMFAA course notes p.13 (the example) what is understood by plane-parallel geometry and is it 1D or 2D? (see answer professor Sundqvist)
\begin{itemize}
\item 
\end{itemize}

\item CMFAA course notes p.15: why is this called diffusion $F=T^3 \frac{dT}{dx}$ (flux proportional to local gradient in temperature)?

\item unclassified: what is the terminal velocity $v_{\infty}$?

\item unclassified: what is Sobo-distribution? (Sobolev distribution)

\item pcyg.f90: for \texttt{test\_number $= 2$}, why do we call it isotropic since isotropy of \texttt{mu} does not imply isotropy of \texttt{theta}? (myself, see definition of intensity)
\end{itemize}

\newpage
\section{Interesting problems}
\begin{itemize}
\item inverse radiative transfer problem
\end{itemize}

\end{document}