\documentclass[../main/main.tex]{subfiles}

\begin{document}

\section{De dagnachtgrens op een wereldkaart}
Het vlak door het middelpunt van de aarde en loodrecht op de zonnestralen heeft vergelijking
\begin{equation}
ax+by+cz = 0
\end{equation}
De punten op de wereldbol die ook doorheen dat vlak liggen worden gegeven door
\begin{equation}
aR\cos(\phi)\cos(\theta) + bR\cos(\phi) \sin(\theta) + cR\sin(\phi) = 0
\end{equation}
\begin{equation}
R^2\cos(\phi)\cos(\theta)\cos(\phi_Z)\cos(\theta_Z) + R^2\cos(\phi) \sin(\theta)\cos(\phi_Z)\sin(\theta_Z) + R^2\sin(\phi)\sin(\phi_Z) = 0
\end{equation}
\begin{equation}
\cos(\phi) \cos(\phi_Z) \cos(\theta - \phi_Z) + \sin(\phi) \sin(\phi_Z) = 0
\end{equation}

\begin{itemize}
\item indien $\phi \neq 0$ dan geldt 
\begin{equation}
\tan(\phi) = -\frac{\cos(\phi_Z) \cos(\theta- \theta_Z)}{\sin(\phi_Z)}
\end{equation}

\item anders geldt dat 
\begin{equation}
\cos(\phi) = 0
\end{equation}
ofwel 
\begin{equation}
\cos(\theta - \phi_Z) = 0
\end{equation}

\end{itemize}

\end{document}