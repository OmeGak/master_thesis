\documentclass[../main/main.tex]{subfiles}
\begin{document}

\section{General equations - first year overview}
This is made in May 2019.

\subsubsection{Hydrodynamics}
Euler equations, together with closing relation (e.g. ideal gas law).

\begin{center}
	\centering
	{\tabulinesep=1.5mm
		\begin{tabu}{|c|c|c|c|} 
			\hline
			\multicolumn{4}{|c|}{primitive variables} \\ \hline
			mass density & velocity & gas energy density & gas pressure \\ 
			$\rho$ & $v$ & $e$ & $p$ \\ \hline
	\end{tabu}}
\end{center}

\subsubsection{Radiation}
Radiative transfer equation: intensity along a ray while interacting with medium. Photons are massless.
\begin{equation}
	\left[ \frac{1}{c}\partial_t + \vec{n}.\vec{\nabla} \right] I_{\nu} =  \eta_{\nu} - \chi_{\nu}I_{\nu}
\end{equation}

\begin{center}
	\centering
	{\tabulinesep=1.5mm
		\begin{tabu}{|c|c|c|c|} 
\hline
frequency & intensity & emissivity & total absorption  \\
$\nu$ & $I_{\nu}$ & $\eta_{\nu}$ & $\chi_{\nu}$ \\ \hline
	\end{tabu}}
\end{center}

These deliver two equations
\begin{itemize}
	\item the radiative energy equation (diffusion flux $\vec{F}$
	\begin{eqnarray}
	\frac{\partial E}{\partial t} + \vec{\nabla} . \vec{F} = \iint ... d\nu d \Omega
	\end{eqnarray}
	\item radiative momentum equation
	\begin{equation}
		\frac{d\vec{F}}{\partial t} = \iint ... \vec{n} d\nu d\Omega
		\end{equation}
\end{itemize}
(after \textbf{integrating over all frequencies}). Depending on the geometry simplifcations, one can e.g. integrate over all solid angles.


\subsubsection{Radiation-Hydrodynamics}
Combination delivers integral-diffusion equation

\begin{eqnarray}
\begin{aligned}
\frac{dI}{d\tau} &=  S - I \\
	&= \int I d\Omega - I
\end{aligned}
\end{eqnarray}

\subsubsection{Challenges}
\begin{itemize}
	\item combination with hydrodynamics
	\item current analysis: simplified geometries (symmetry). E.g. in 2D, an ADI method is used and now also a multigrid method. 
	\item complex geometry difficult to show in ray-tracing scheme
	\item steady-state vs. time dependent
	\item focus on radiation equations
\end{itemize}

\end{document}