\documentclass[../main/main.tex]{subfiles}
\begin{document}

\paragraph{forewarned is forearmed}
General guidelines for good practices in scientific computing are found in \cite{Wilson2014}.

\section{Introduction to Monte Carlo Radiation Transfer}
The material is taken from \cite{WWBW2001} and from \cite{WWBW2013}.

\subsection{Elementary principles}

\begin{center}
\centering
{\tabulinesep=1.5mm
\begin{tabu}{|c|c|}
\hline 
specific intensity & $I_{\nu}$ \\ \hline
radiant energy & $dE_{\nu}$ \\ \hline
surface area & $dA$ \\ \hline
angle & $\theta$ \\ \hline
solid angle & $d \Omega$ \\ \hline
frequency range & $d \nu$ \\ \hline
time & $dt$ \\ \hline
flux & $F_{\nu}$ \\ \hline
cross section & $\sigma$ \\ \hline
scattering angle & $\chi$ \\ 
 & $\mu = \cos(\chi)$ \\ \hline
mean intensity & $J$ \\ \hline
flux & $H$ \\ \hline
radiation pressure & $K$ \\ \hline
\end{tabu}}
\end{center}

\subsection{Example: plane parallel atmosphere}
\begin{enumerate}

\item emission of photons: select two angles (3D space). In isotropic scattering
\begin{itemize}
\item $\theta$ met $\mu = \cos(\theta)$
	\begin{itemize}
	\item $\mu = 2\xi -1$ (isotropic scattering)
	\item $\mu = \sqrt{\xi}$ (A slab is heated from below. Then $P(\mu) = \mu$)
	\end{itemize}
\item $\phi = 2 \pi \xi$
\end{itemize}

\item propagation of photons
\begin{itemize}
\item sample optical depth from $\tau = -\log(\xi)$
\item distance travelled $L = \frac{\tau z_{max}}{\tau_{max}}$
\end{itemize}

\item conclusion of emission and propagation
\begin{equation}
\begin{aligned}
x &= x + L \sin(\theta) \cos(\phi) \\
y &= y + L \sin(\theta) \sin(\phi) \\
z &= z + L \cos(\theta)
\end{aligned}
\end{equation}

\item Binning: once the photon exists the slab. Produce histograms of the distribution function. Finally, we wish to compute the output flux or the intensity.
\end{enumerate}

\newpage
\section{Asymptotic Preserving Monte Carlo methods for transport equations in the diffusive limit}
A very interesting article about Monte Carlo methods for radiative transfer problems, from a mathematical point of view, is \cite{Dimarco2018}. I am currently trying to reproduce the numerical experiments that are reported in the article. 

\section{Fluid and hybrid Fluid-Kinetic models (for neutral particles in plasma edge) (Horsten2019)}
The material is mainly taken from \cite{HorstenNiels2019}.
\begin{itemize}
\item Kinetic Boltzmann equation: neutral velocity distribution $f_n(r,v)$
\item If you taken into account (e.g. microscopic processes for atomic deuterium) then the kinetic Boltzmann equation becomes 
\begin{equation}
v\nabla f_n(r,v) = S_r(r,v) + S_{cx}(r,v) - f_n(r,v)(R_{cx}(r,v)+R_i(r))
\label{Horsten_kinetic_equation}
\end{equation}

\item Numerical solution strategies
\begin{itemize}
\item finite differences/volumes/elements :computationally infeasible
\item spectral methods (series expansion of $f_n(r,v)$: not suitable for modelling discontinuties
\item stochatic approach: the whole velocity distribution is discretized by finite set of particles
\end{itemize}  

\item from Equation (\ref{Horsten_kinetic_equation}), the fluid model and the hybrid model is derived.
\begin{itemize}
\item Fluid model: 3 state equations (continuity -  momentum - energy) with boundary conditions
\begin{itemize}
\item pure-pressure equation: maximum error of 10 - 28 \% 
\item with parallel momentum source: error 10 \%
\item with ion energy source: error 30 \% 
\end{itemize} 

\item hybrid model based on micro-macro decomposition
\end{itemize}
\end{itemize}

\newpage
\section{Overview of existing (Monte Carlo) radiative transfer codes}

\subsection{Synthesis codes}
As is pointed out in \cite{PinteChristophe2015CRT}, there are basically two methods to solve the radiative transfer problem: ray-tracing and Monte Carlo methods.

\begin{itemize}
\item RADICAL \cite{RADICAL}  (Ray-tracing, 2D, multi-purpose)
\item MULTI \cite{MULTI} \cite{Carlsson1986} (computer program for solving multi-level non-LTE radiative transfer problems in moving or static atmospheres, very old: Uppsala 1986 - 1995)
\item SKIRT \cite{Camps2015} (continuum (Monte Carlo) radiation transfer in dusty astrophysical systems, such as spiral galaxies and accretion disks, from Ugent)
\item TORUS \cite{Harries2019} (Monte Carlo radiation transfer and hydrodynamics code. Adopts 1D, 2D, 3D adaptive mesh refinement. Suitable for radiative equilibrium and creation of synthetic images and SED)
\item RADMC-3D \cite{RADMC3D} (Monte Carlo code that is especially applicable for dusty molecular clouds, protoplanetary disks, circumstellar envelopes, dusty tori around AGN and models of galaxies. Python interface with Fortran main code)
\item TLUSTY and SYNSPEC \cite{Hubeny2017}, \cite{Hubeny2017a}, \cite{Hubeny2017b}.
\end{itemize}

\subsection{Inversion codes}
\begin{itemize}
\item VFISV
\item ASP/HAO
\item HeLIx+
\item SNAPI (not publicly available, created by Ivan Milic)
\item multiple codes available from Instituto de Astrofyiica de Canarias (IAC)
\item STiC: the Stockholm inve rsion code
\end{itemize}

\end{document}

